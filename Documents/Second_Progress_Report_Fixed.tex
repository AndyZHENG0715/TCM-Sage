\documentclass[12pt,a4paper]{article}

% Packages
\usepackage[utf8]{inputenc}
\usepackage[margin=1in]{geometry}
\usepackage{parskip}
\usepackage{titlesec}
\usepackage{booktabs}
\usepackage{array}
\usepackage{enumitem}
\usepackage{hyperref}
\usepackage{float}  % Added to force table placement

% Table column type for wrapped text
\newcolumntype{L}[1]{>{\raggedright\arraybackslash}p{#1}}

% Section formatting
\titleformat{\section}{\large\bfseries}{\thesection.}{0.5em}{}
\titleformat{\subsection}{\normalsize\bfseries}{\thesubsection.}{0.5em}{}

\begin{document}

\begin{center}
    \Large\textbf{COMP4908 Final Year Project: Second Progress Report}
\end{center}

\noindent\textbf{Project Title:} TCM-Sage: An Evidence-Synthesis Tool for Traditional Chinese Medicine\\
\textbf{Student:} ZHENG Zian (22231153)\\
\textbf{Supervisor:} ZHANG Ce\\
\textbf{Date:} December 31, 2025\\
\textbf{Reporting Period:} October 2025 -- December 2025 (Phase 2)

\vspace{1em}
\noindent\rule{\textwidth}{0.4pt}
\vspace{1em}

\section{Executive Summary}

This report outlines the significant progress achieved during Phase 2 (MVP Implementation) of the ``TCM-Sage'' project. Following the foundational research in Phase 1, the primary objective of this period was to develop a functional \textbf{Minimum Viable Product (MVP)} capable of end-to-end Retrieval-Augmented Generation (RAG).

The project successfully transitioned from theoretical architecture to working software. The system now features a fully operational \textbf{Hybrid Retrieval} engine (combining semantic search with a knowledge graph) and a \textbf{Reflective Generator} that implements real-time query routing and self-correction to ensure safety.

All planned milestones for the MVP have been met, and the system is ready for demonstration. This report also addresses previous supervisor feedback regarding schedule visualization and specific evaluation metrics.

\section{Detailed Achievements in Phase 2}

\subsection{Core RAG Pipeline \& Multi-Provider Architecture}

The fundamental ``nervous system'' of TCM-Sage has been implemented in Python. The system ingests user queries, retrieves relevant context from the \textit{Huangdi Neijing} (Yellow Emperor's Inner Canon), and synthesizes evidence-based answers.

\begin{itemize}[leftmargin=*]
    \item \textbf{Vector Store Integration:} The text was programmatically cleaned, split, and embedded using \texttt{all-MiniLM-L6-v2} into a persistent \textbf{ChromaDB} vector store, enabling sub-second semantic retrieval.
    \item \textbf{Multi-Provider Flexibility:} To ensure cost-efficiency and testing rigor, the system features a provider-agnostic layer. It can seamlessly switches between \textbf{Alibaba Cloud} (Qwen-3-max) for heavy testing and \textbf{Google} (Gemini-3-Flash) for high-fidelity benchmarking via environment variable configuration.
\end{itemize}

\subsection{Architectural Pivot: Hybrid Retrieval Engine}

A major technical achievement this phase was the implementation of a \textbf{Hybrid Retriever}. Early testing revealed that pure vector search struggled with specific herb-symptom relationships (e.g., distinguishing between a ``headache'' in a metaphor versus a clinical treatment).

\begin{itemize}[leftmargin=*]
    \item \textbf{Ensemble Context Aggregation:} Instead of complex score merging, the system employs an ``Ensemble Context'' strategy. It retrieves two distinct streams of data:
    \begin{enumerate}
        \item \textbf{Semantic Text Chunks (Vector):} For broad context and theoretical explanations.
        \item \textbf{Structured Facts (Knowledge Graph):} For precise entity relationships (e.g., \textit{Headache --- TREATS --- Chuanxiong}).
    \end{enumerate}
    \item \textbf{``Glass Box'' Reasoning:} These streams are presented to the LLM as distinct sections (\texttt{=== Text Passages ===} and \texttt{=== Knowledge Graph Facts ===}). This allows the LLM to explicitly cite structured facts, significantly improving the precision of prescriptive answers.
\end{itemize}

\subsection{The Reflective Generator: Safety \& Self-Correction}

Given the medical nature of the project, ``hallucination'' is a critical safety risk. To mitigate this, a \textbf{Reflective Generator} architecture (inspired by the ``Self-RAG'' framework) was implemented.

\begin{itemize}[leftmargin=*]
    \item \textbf{Intelligent Query Routing:} The system classifies every incoming query as either \textbf{``Informational''} (general concepts, Temp 0.7) or \textbf{``Prescriptive''} (clinical advice, Temp 0.0).
    \item \textbf{Self-Correction Loop:} A post-generation verification module freezes the output and prompts a secondary lightweight model to audit the response. If unsupported claims are detected, a warning flag (\texttt{! [Self-Critique Warning]}) is automatically appended, ensuring users are alerted to potential inaccuracies in real-time.
\end{itemize}

\section{Challenges Encountered \& Solutions}

\subsection{Ambiguity in Classical Terminology}

\textbf{Challenge:} The \textit{Huangdi Neijing} uses archaic terminology where a single character can imply a symptom, an organ, or a cosmological concept. Standard embeddings often failed to capture these precise nuances.

\textbf{Solution:} The shift to \textbf{Hybrid Retrieval}. By mapping key clinical entities (Symptoms, Herbs) into a graph structure, we provided a ``ground truth'' pathway. Even if semantic search misses a subtle connection, the explicit graph edge guarantees retrieval.

\subsection{Knowledge Graph Construction Strategy}

\textbf{Challenge:} The original plan called for fully automated Information Extraction (IE) in Phase 2. However, initial experiments showed that automated extraction on Classical Chinese text produced significant noise.

\textbf{Solution:} We prioritized \textbf{Reliability over Scale} for the MVP. A \textbf{Manual JSON Curation} strategy was adopted for the Phase 2 demo, verifying a ``Golden Path'' dataset focusing on common ailments (e.g., Headache, Insomnia). The automated extraction pipeline has been rescheduled to Phase 3 as the primary focus.

\section{Evaluation Metrics}

To objectively measure the system's effectiveness in Phase 4, the following Key Performance Indicators (KPIs) have been defined:

\begin{table}[H]
\centering
\begin{tabular}{@{}L{3.5cm}L{4cm}L{6.5cm}@{}}
\toprule
\textbf{Metric Category} & \textbf{Metric Name} & \textbf{Target / Method} \\
\midrule
\textbf{Performance} & \textbf{Response Latency} & \textbf{Less than 5 seconds} for the full retrieval \& generation cycle. \\
\textbf{Accuracy} & \textbf{Citation Precision} & \textbf{More than 90\%} of generated claims must link to the correct source text (Human Evaluation). \\
\textbf{Safety} & \textbf{Hallucination Rate} & \textbf{Less than 10\%} on prescriptive queries (Verified by domain expert audit). \\
\textbf{User Experience} & \textbf{Perceived Trust} & Likert Scale (1-5) collected during pilot testing. Target: \textbf{More than 4.0}. \\
\bottomrule
\end{tabular}
\end{table}

\clearpage
\section{Upcoming Schedule (Phase 3 \& 4)}

The focus for the next period shifts from ``Core Architecture'' to ``Automation and Evaluation.''

\begin{table}[H]
\centering
\small
\begin{tabular}{@{}L{2.5cm}L{2.5cm}L{9cm}@{}}
\toprule
\textbf{Phase} & \textbf{Duration} & \textbf{Key Tasks \& Milestones} \\
\midrule
\textbf{Phase 3: Enhancement} & \textbf{Jan 10 -- Feb 10} & \textbf{1. Automated KG Construction:} Develop Python scripts using LLMs to parse the 81 chapters of \textit{Su Wen}, scaling the graph from dozens to thousands of nodes.\newline\textbf{2. Web Interface:} Migrate from CLI to \textbf{Streamlit} Web UI.\newline\textbf{3. Explainability:} Implement ``Reasoning Visualization'' UI to show users exactly which text chunks/graph nodes were used. \\
\midrule
\textbf{Phase 4: Evaluation} & \textbf{Feb 11 -- Mar 10} & \textbf{1. Quantitative Benchmarking:} Run the ``Golden Set'' of 20 clinical questions against the metrics defined in Section 4.\newline\textbf{2. Domain Expert Validation:} Conduct pilot testing with \textbf{3-5 students/practitioners} from the HKBU School of Chinese Medicine to validate clinical relevance and usability. \\
\midrule
\textbf{Phase 5: Finalizing} & \textbf{Mar 11 -- Apr 8} & \textbf{1. Final Report Writing:} Compile all findings and technical documentation.\newline\textbf{2. Presentation Prep:} Prepare poster and slides for the April oral defense. \\
\bottomrule
\end{tabular}
\end{table}

\section{Conclusion}

Phase 2 has been highly productive. The transition from a theoretical proposal to a working codebase involves overcoming significant complexity in retrieving information from classical texts. By implementing the Hybrid Retriever and Self-Correction module, TCM-Sage has evolved into a safety-conscious, evidence-backed assistant. I am confident in the current state of the system for the upcoming mid-point presentation and prepared for the scaling challenges in Phase 3.

\end{document}
